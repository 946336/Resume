%%%%%%%%%%%%%%%%%%%%%%%%%%%%%%%%%%%%%%%%%
% Medium Length Professional CV
% LaTeX Template
% Version 2.0 (8/5/13)
%
% This template has been downloaded from:
% http://www.LaTeXTemplates.com
%
% Original author:
% Trey Hunner (http://www.treyhunner.com/)
%
%%%%%%%%%%%%%%%%%%%%%%%%%%%%%%%%%%%%%%%%%

%-------------------------------------------------------------------------------
%   PACKAGES AND OTHER DOCUMENT CONFIGURATIONS
%-------------------------------------------------------------------------------

\documentclass{resume} % Use the custom resume.cls style

% Document margins
\usepackage[left=0.75in,top=0.6in,right=0.75in,bottom=0.3in]{geometry}
\usepackage{hyperref}

\name{Wesley Wei} % Your name
% \address{新竹市科學園路107向8號5樓之1} % Your address
% \address{40 Lower Campus Rd, Latin Way F272} % Your address
%(optional)
% Your phone number and email address
\address{617-275-9957 \\
    Wesley@WeiWesley.com \\
    U.S. Citizen
% GitHub: 946336
}

\begin{document}

%-------------------------------------------------------------------------------
%   TECHNICAL STRENGTHS SECTION
%-------------------------------------------------------------------------------

\begin{rSection}{Technical Strengths}

\begin{tabular}{ @{} >{\bfseries}l @{\hspace{6ex}} l }
Programming Languages & C, C++, Python, Bash/Shell, C\#, HTML/CSS/Javascript \\
Technologies \& Tools & GNU/Linux, Vim, Git \\
\end{tabular}

\smallskip
% \bigskip
\end{rSection}

%-------------------------------------------------------------------------------
%   EDUCATION SECTION
%-------------------------------------------------------------------------------

\begin{rSection}{Education}

\textbf{Tufts University, Medford, MA}, \textit{Class of 2018} \\
Bachelor of Science, Computer Science \\
Bachelor of Science, Mathematics
% GPA 3.29/4.0 \smallskip
% {\em {\bf \em Relevant coursework}: Data Structures, Machine Structure \&
% Assembly, Programming Languages, Algorithms, Advanced Functional Programming,
% Operating Systems, Machine Learning, Computer Graphics, Computational Theory,
% Programming Language Design.}
% \smallskip

\smallskip
% \bigskip
\end{rSection}

%-------------------------------------------------------------------------------
%   WORK EXPERIENCE SECTION
%-------------------------------------------------------------------------------

% Resume section for work experience, etc
\begin{rSection}{Experience}

% Subsection for each entry
% \begin{rSubsection}{Employer, location, etc}{Date(s)}{Job title}
% \item TEXT
% \end{rSubsection}

\begin{rSubsection}{Tufts University, Medford, MA}{September 2015 -
    May 2018}{Teaching Assistant - Department of Computer Science}

\item Answered students' questions regarding assignments and topics covered in
    class.

\item Graded student design documents and provided feedback on document
    structure, as well as commenting on program design when necessary.

\end{rSubsection}

% \begin{rSubsection}{NSRRC, Hsinchu, Taiwan}{June 2016 - August 2016}{Temporary
%     Assistant}

% \item Refactored a Python/PyQt based frontend for a Linux based image
%     processing backend with the goal of easing future development and the
%     addition of new features.

% \item Added the capability to operate on subsets of data, letting the
%     application to load a fraction of thousands of images and operate much
%     faster.

% \end{rSubsection}

% \begin{rSubsection}{NTHU, Hsinchu, Taiwan}{June 2015 - November
%     2015}{Temporary Assistant}

% \item Wrote a MATLAB script automating data analysis, replacing manual
%     calculations involving ORIGIN and Microsoft Excel, cutting processing time
%     from 20 minutes to a few seconds per data set.

% \end{rSubsection}

\smallskip
% \bigskip

\end{rSection}

%-------------------------------------------------------------------------------
%   OTHER ACTIVITIES
%-------------------------------------------------------------------------------

\begin{rSection}{Projects}

\begin{rSubsection} {Time Series Visualization Utility}
    {August 2018 - September 2018}{}

    \item Used C\# and DevExpress to create a program to graph time series data
        and export charts as images with precisely controlled dimensions.

    \item Designed a data format based on CSV to carry both data and display
        metadata, including color, line weight, and line style.

\end{rSubsection}

\begin{rSubsection}{PDF Report Generator}{October 2018}{}

    \item Automated report generation with Python, pypyodbc, and PyFPDF. Reports
        include numerous charts produced using Matplotlib.

\end{rSubsection}

\begin{rSubsection}{Dask Cluster Setup}{October 2018}{}

\item Automated setup of Linux virtual machines intended for parallel computing
    using Dask using shell scripts and cloud-init.

\end{rSubsection}

\begin{rSubsection}
    {REPL}
    {June 2018 - September 2018}{Hosted on Github}

\item REPL is a python framework for embedding a shell into an application,
    providing a simple way to bind python code to textual commands.

\item REPL mimics some of the conveniences of a POSIX shell, providing pipes,
    functions, flow control, aliases, and more, allowing developers and end
    users to quickly create larger programs by composing existing commands.

\item REPL supports configuration through startup dotfiles.

\end{rSubsection}

\begin{rSubsection}
    {Composte}
    {Fall 2017}{Hosted on GitHub}

\item Composte is written in Python and is a Linux based client-server
    application that facilitates real-time, collaborative editing of sheet
    music.

\item Implemented network foundation using ZeroMQ, serverside storage using
    SQLite, an ad hoc shell for user interaction and scripting, and encryption
    stubs.

\item Composte includes a PyQt GUI developed by one team member and a music
    backend developed by another team member, based on the music21 package.

\end{rSubsection}

\begin{rSubsection}
    % {Questionable Battleship}
    {Questionable Battleship}
    {Summer 2017}{Hosted on GitLab}

% \item Multiplayer Battleship over websockets. Available at
%     questionablebattleship.com/simple.

\item Developed simple web-based UI using Javascript and jQuery.

\item Developed backend server in C++, relying on the following libraries:
    crossguid, doctest, json, spdlog, tclap, websocketpp, cppzmq,
    libicu. The server uses worker processes to manage its state and ZeroMQ
    for IPC.

\end{rSubsection}

% \begin{rSubsection}
%   {\href{https://www.cap-tal.com} {Cap-Tal}}
%   {Summer 2017 - Present}{}

% \item Built a simple website for a small, loose development team under the
    % name Cap-Tal.

% \end{rSubsection}

% \begin{rSubsection}
%     {Polyhack at Tufts}
%     {October 2016}{Hosted on GitHub}

% \item Snapsassin is a Java-based Android app using facial recognition to play
%     the game "Assassin." Game state is stored in the cloud in a Firebase
%     database.

% \item Used Volley to implement wrapper methods around Microsoft's Facial
%     Recognition API in order to determine whether or not the correct player
%     appeared in photos.

% % \item Android application to automate moderation of the game Assassin, using a
% %     face recognition API to confirm “kills.”
% \end{rSubsection}

% \begin{rSubsection}
%     {ID Hack at Tufts}
%     {March 2016}{Team HandSPASMS (Qualcomm Award)}

% \item Worked with a team to develop an Android application to enable doctors
%     to easily remind patients of upcoming appointments via SMS

% \item Diagnosed HTTP request handling errors in the application

% \end{rSubsection}

\end{rSection}

% \begin{rSection}{Other Activities}

% \begin{rSubsection}{}{}{}

% \end{rSubsection}

% \begin{rSubsection}{Proofreading Academic Manuscripts}{2010 - Present}{}
% \item Proofread various articles for researchers from NSRRC and NTHU
% \end{rSubsection}

% \smallskip
% \bigskip

% \end{rSection}

%-------------------------------------------------------------------------------
%   EXAMPLE SECTION
%-------------------------------------------------------------------------------

%\begin{rSection}{Section Name}

%Section content\ldots

%\end{rSection}

%-------------------------------------------------------------------------------

\end{document}

